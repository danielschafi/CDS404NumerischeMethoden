\section{Code Snippets}

\subsection{numpy}

\paragraph{import}

\begin{minted}{python}
import numpy as np
\end{minted}

\subsubsection{linspace}

\begin{minted}{python}
np.linspace(start, stop, num=50)
\end{minted}

\subsection{matplotlib}

\subsubsection{import}

\begin{minted}{python}
import matplotlib.pyplot as plt
\end{minted}

\subsubsection{rcParams}

Mit den \textbf{rcParams} kann die visualisierung beliebig konfiguriert werden.

\begin{minted}{python}
plt.rcParams['figure.figsize'] = (7.03, 15)
plt.rcParams['font.size'] = 9
plt.rcParams['font.family'] = 'serif'
plt.rcParams['text.usetex'] = True
\end{minted}

\subsubsection{plot}

\begin{minted}{python}
plt.plot(x_data, y_data, '--', linewidth=3, label=r'$g$')
plt.xlabel(r'$x$')
plt.ylabel(r'$y$')
plt.legend()
plt.grid(visible=True)
plt.axis('image')
\end{minted}

\subsection{scipy}

\subsubsection{import}

\begin{minted}{python}
import scipy.interpolate as ip
\end{minted}

\subsubsection{Polynom interpolation}

\begin{minted}{python}
po = ip.BarycentricInterpolator(x_data, y_data)
\end{minted}

\subsubsection{Cubic Spline}

\begin{minted}{python}
cs = ip.CubicSpline(x_data, y_data)
\end{minted}