\section{Numerische Integration}

Ist die Stammfunktion nicht bekannt, so kann das Integral durch eine
einfachere funktion approximiert werden.

Dabei wird die Annäherung durch die folgenden Funktionen realisiert.

\begin{itemize}
    \item konstante Funktion
    \item lineare Funktion
    \item quadratische Funktion
\end{itemize}

\subsection{Konstante Funktion}

\paragraph{Links}

\begin{displaymath}
    I = h \sum_{k=0}^{n-1} f(x_k)
\end{displaymath}

\paragraph{Rechts}

\begin{displaymath}
    I = h \sum_{k=1}^{n} f(x_k)
\end{displaymath}

\paragraph{Mittelpunkt}

\begin{displaymath}
    I = h \sum_{k=0}^{n-1} f \biggl (\frac{x_k + x_{k+1}}{2} \biggr )
\end{displaymath}

\subsection{Lineare Funktion (Trapez Regel)}

\paragraph{Allgemein}

\begin{displaymath}
	I = \frac{1}{2} \sum_{k=0}^{n-1} (x_{k+1} - x_k) \cdot (f(x_{k+1}) + f(x_k))
\end{displaymath}

\paragraph{Äquidistante X-Werte}

\begin{displaymath}
	I = \frac{h}{2} \sum_{k=0}^{n-1} (f(x_{k+1}) + f(x_k))
\end{displaymath}

\subsection{Quadratische Funktion (Simpson Regel)}

\begin{displaymath}
	A_k = \frac{h}{3} (f(x_{k-1}) + 4f(x_k) + f(x_{k-1}))
\end{displaymath}